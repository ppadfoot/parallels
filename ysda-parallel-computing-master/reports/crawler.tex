\documentclass{article}

\usepackage[utf8]{inputenc}
\usepackage{amsmath, amssymb}
\usepackage[english,russian]{babel}
\usepackage[T1]{fontenc}
\usepackage[left=1.5cm,right=1.5cm,top=2cm,bottom=2cm,bindingoffset=0cm]{geometry}
\usepackage{tikz}
\usepackage{pgfplots}

\begin{document}

    \textbf {Выбранный подход}

    Был выбран подход, позволяющий распараллелить выполнение задачи без использования разделяемых ресурсов и локов.
    Отстутствие локов и разделяемых переменных (а следовательно и отсутствие false sharing'а и data race'ов) значительно
    ускоряет выполнение программы.
    Плюс данного подхода еще в том, что он значительно упрощает написанную программу.

    Идея заключается в том, чтобы распараллелить каждую волну поиска в ширину.
    То есть каждая волна поиска начинается с того, что мы аггрегируем результаты предыдущей, фильтруем только те,
    которых в $HashSet<string> visited$ не было, кладем их в $visited$ и запускаем параллельно проход по всем этим ссылкам.
    При этом заметим, что нам не нужно при каждом выполнении скачивания страницы смотреть в $visited$, и брать дорогой
    лок на эту операцию.

    Также выигрыш в производительности достигается за счет того, что в веб графах обчыно очень много ссылок, поэтому на
    каждой волне поиска в ширину можно выиграть много времени при распараллеливании.

    Для распараллеливания загрузки страниц используется метод C\# $Parallel.ForEach$.
    В него можно передать максимальное количество потоков, которые он может задействовать или передать -1,
    чтобы он он сам подобрал оптимальное количество потоков.
    Под капотом этот метод использует встроенную библиотеку $Task Parallel Library (TPL)$, а конкретно механизм пула потоков.
    Метод $Parallel.ForEach$ также определяет оптимальное количество Task'ов (то есть он не будет ставить 100 тысяч тасок, а объединит их в батчи и уменьшит тем самым накладные расходы).

    \textbf {Тестирование производительности}

    Производительность тестировалась на следующих настройках: crawler.exe \ "http://yandex.ru 2 100 crawl\ "
    При тестировании алгоритм запускался 10 раз, затем считалось среднее и дисперсия.
    Получились следующие результаты (результат в секундах, в скобках указана дисперсия):

    \begin{enumerate}
        \item WebCrawlerSync(): Elapsed: 55.27305424 (6.22984472886608)
        \item WebCrawlerAsync(-1): Elapsed: 22.89990244 (4.26477578656015)
        \item WebCrawlerAsync(1): Elapsed: 55.94903001 (1.82260854321946)
        \item WebCrawlerAsync(2): Elapsed: 34.44340731 (3.26883298908731)
        \item WebCrawlerAsync(4): Elapsed: 24.92590704 (4.02489514486327)
        \item WebCrawlerAsync(8): Elapsed: 20.08012662 (3.41057027656706)
        \item WebCrawlerAsync(16): Elapsed: 19.41109792 (2.9251830158163)
    \end{enumerate}

    Здесь WebCrawlerSync() - полностью синхронная реализация, WebCrawlerAsync(n), где n - количество потоков, - распараллеленая реализация

    Также были произведены запуски на других сайтах с большим количеством страниц с такими настройками: \ "crawler.exe http://google.com 3 1000 crawl\_google\ " со следующими результатами:

    \begin{enumerate}
        \item WebCrawlerSync():    Elapsed: 806,3208642 (13,0145307000025)
        \item WebCrawlerAsync(-1): Elapsed: 119,30601245 (7,22637694999998)
    \end{enumerate}

    Из этого видно, что намного больший выигрыш в производительности можно получить на большом количестве страниц.
\end{document}